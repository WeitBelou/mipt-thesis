\chapter{Введение}\label{ch:intro}

\section{Что за задача?}\label{sec:intro-problem}

Решаем задачу пробивания тканевого экрана облаком высокоскоростных оскольков.

Рассматриваются:

\begin{enumerate}
    \item Аналитические соотношения
    \item Модель сплошной среды
    \item Модель тонкой мембраны
    \item Численный метод сглаженных частиц (SPH)
    \item Система уравнений для системы тонких нитей
\end{enumerate}

Полученные численные результаты сопоставляются с аналитическими решениями и экспериментальными данными.

Описан метод моделирования на основании системы уравнений для тонких нитей,
приведены результаты рассчётов, а также  их сравнение с аналитическими решениями и экспериментальными данными.
Получено хорошее соответствие между рассчётами, аналитикой и экспериментом.
