\section{Постановка задачи}\label{sec:intro-problem}

В работе рассматривается задача о пробивании тканевого композитного экрана облаком высокоскоростных осколков. 
Тканевый экран представляет собой элемент защитной оболочки космического аппарата. Защитная оболочка в полной 
конфигурации представляет из себя разнесённую преграду -- первый тонкий металлический слой обеспечивает дробление 
потенциально опасной частицы, второй тканевый композитный слой должен минимизировать воздействие её осколков на 
защищаемую стенку. Реальный тканевый экран состоит из разнотипных слоёв, часть из которых обеспечивает 
защиту от силового воздействия скоростных осколков, а часть -- от теплового потока. В данной работе 
рассматривается только силовая компонента воздействия.

Параметры волокон силовой ткани приведены в \tabref{tbl:kevlar-params}.

\begin{table}[h]
    \centering
    \caption{Параметры ткани}\label{tbl:kevlar-params}
    \begin{tabular}{|l|l|}
        \hline
        Параметр & Значение      \\ \hline
        Плотность & 1430 $кг/м^3$ \\ \hline
        Модуль Юнга & 125 ГПа       \\ \hline
        Предельное удлинение при разрыве & 4.25 \%       \\ \hline
        Прочность на разрыв & 4 ГПа         \\ \hline
        Прочность на сжатие & $40-60$ МПа     \\ \hline
    \end{tabular}
\end{table}

Данные по модулю Юнга, удлинению при разрыве, прочности на разрыв взяты из\cite{perepelkin2009,mikhailin2013}.
Значение прочности на сжатие взято, на основании данных\cite{papkov1986} о том, что прочность параамидных волокон
на сжатие $\approx 1.0 - 1.5 \%$ от прочности на растяжение.

Исходно внешнее воздействие представляет собой одиночную частицу (метеор, космический мусор) массой несколько грамм, 
двигающуюся со скоростью несколько километров в секунду. Однако, после взаимодействия частицы с первым металлическим 
экраном формируется облако мелких осколков (как самой частицы, так и материала экрана). Именно данное облако воздействует 
на тканевый экран. При этом определение точной структуры данного облака является отдельной достаточно сложной задачей. 
В данной работе в качестве внешнего воздействия рассматривается эквивалентный импульс, заданный на поверхности экрана.

В работе рассматривается ряд подходов к решению задачи: 
аналитические инженерные соотношения, 
модель сплошной среды и численный метод сглаженных частиц, 
модель тонкой мембраны, 
модель системы тонких нитей.

