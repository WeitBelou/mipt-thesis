\chapter{Обзор существующих работ}\label{ch:overview}

На данный момент для задачи пробития тканной композитной брони наиболее распространена следующая постановка:
массивный ударник на скоростях равных сотням $м/с$ ударяется об один, или несколько слоёв материала.
Часто в экспериментах располагают дополнительно слой из пластичного материала для оценки повреждения наносимого
остаточным импульсом ударника~\cite{kobylkin2014}.
Данная конфигурация соответствует попаданию пули выпущенной из стрелкого оружия по лёгкой тканной нательной броне.

Однако диапазон скоростей и конфигурация нагрузки, возникающие в задачи поражения нательной брони пулями, выпущенными
из стрелкового оружия существенно отличается от таковых в случае воздействия облака космического мусора на защитные
композитные экраны космических аппаратов.

Однако диапазон скоростей, рассматриваемый в нашей задаче существенно отличается от рассматриваемых ранее ($> 1 км/с$).
Также существенно отличается конфигурация нагрузки: облако мелких осколков в отличии от одного массивного ударника.
Ранее рассчёты подобной задачи проводились в~\cite{farenthold}.
