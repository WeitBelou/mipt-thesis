\section{Обзор существующих работ}\label{ch:overview}

Вопросы расчетно-экспериментального исследования прочности защиты элементов ракетно-космической техники при
высокоскоростном воздействии частиц осколочно-метеороидной среды получили интенсивное развитие во второй
половине ХХ века.

Однако, применение тканевых композитов из волокнистых материалов для создания систем защиты космических аппаратов
является существенно новой областью, практически не изученной на сегодняшний день.
Принципиальным фактором в данной задаче оказывается скорость воздействующих частиц (5-7 км/с).

На данный момент для задачи пробития тканой композитной брони наиболее распространена постановка
~\cite{kobylkin2014,kharchenko,walker1999,walker2001,porval,bkhatnagar}, соответствующая попаданию пули, 
выпущенной из стрелкого оружия, по лёгкой тканой нательной броне.
Применимость данных методов для задачи пробивания экрана облаком малых осколков на скорости несколько километров в
секунду как минимум  требует отдельного изучения и значительной адаптации методик расчёта.

При выборе математической модели и методики расчёта следует  учитывать, что, несмотря на обширный накопленный опыт
проектирования и  эксплуатации композитных деталей, единой теории разрушения композитов,  достаточно надежной для
практического применения в случае произвольных  параметров нагружения, на данный момент не существует.
В работе~\cite{kaddour} приведен обзор результатов проекта WWFE II, в рамках которого  множеством научных коллективов со всего
мира проводилось сравнение ряда  современных критериев разрушения композитных материалов друг с  другом и результатами
натурных экспериментов.
По итогам делается  заключение, что универсальный критерий ещё не разработан.

Отдельной проблемой является тот факт, что в случае динамического  нагружения информация о прочностных свойствах
материалов носит разрозненный характер~\cite{kobylkin2014}.
В результате, как правило, при математическом моделировании динамических задач применяются те же параметры для
критериев разрушения, что и при статических расчётах.
Соответственно, значения порогов разрушения также определяются при экспериментах по  статическому нагружению, что
является одной из причин расхождения численных экспериментов с натурными.

В связи с указанными сложностями требуется как проведение натурных экспериментов в интересующем скоростном
диапазоне, так и разработка специализированной математической модели с учётом особенностей данной задачи.
