\chapter{Аналитические выражения}\label{ch:equations}
В данном разделе выполнены аналитические оценки для рассматриваемой задачи пробивания листа обаком осколков.
Оценки проведены по аналогии с пробитием бронежилетов пулями.
Важная особенность состоит в том, что существующие рассчёты проведены в иных скоростных режимах (скорость пули менее
1 $км/c$ ) и при существенно ином пространственном распределении нагрузки (массивный ударник, а не облако мелких
осколков).
Не смотря на это, после некоторой колибровки и сопоставления с экспериментом и численными рассчётами, результаты
аналитических вычислений можно использовать чтобы далее улучшить точность аналитических решений.

В этом разделе рассматривается композитный экран из множества слоёв ткани.
Параметры волокон ткани приведены в \tabref{tbl:kevlar-params}

\begin{table}[h]
    \centering
    \caption{Параметры ткани}\label{tbl:kevlar-params}
    \begin{tabular}{|l|l|}
        \hline
        Параметр & Значение      \\ \hline
        Плотность & 1430 $кг/м^3$ \\ \hline
        Модуль Юнга & 125 ГПа       \\ \hline
        Предельное удлинение при разрыве & 4.25 \%       \\ \hline
        Прочность на разрыв & 4 ГПа         \\ \hline
        Прочность на сжатие & $40-60$ МПа     \\ \hline
    \end{tabular}
\end{table}

Данные по модулю Юнга, удлинению при разрыве, прочности на разрыв взяты из\cite{perepelkin2009}