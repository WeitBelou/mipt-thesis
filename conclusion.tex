\section{Заключение}\label{ch:conclusion}

В работе был рассмотрен ряд подходов к решению задачи о пробивании тканевого композитного экрана облаком высокоскоростных осколков:
\begin{enumerate}
    \item аналитические инженерные соотношения;
    \item модель сплошной среды и численный метод сглаженных частиц;
    \item модель тонкой мембраны;
    \item модель системы тонких нитей.
\end{enumerate}

По итогам рассмотрения для реализации была выбрана модель системы тонких нитей,
так как данная модель позволяет учесть особенности поставленной задачи.
Данная модель была реализована в виде программы на языке Python.

С использованием реализованной модели были выполнены расчёты деформирования тканевого экрана.
В качестве внешнего воздействия рассматривался эквивалентный импульс, заданный на поерхности экрана.
Были выполнены серийные расчёты для различных параметров экрана и импульса нагрузки.
Получена форма разрушения экрана в виде креста, наблюдаемая в натурных экспериментах.

Развитием данной работы может являться доработка модели для учёта следующих эффектов:
\begin{enumerate}
    \item влияние соударения и взаимного трения нитей на картину деформации;
    \item учёт оборванных нитей при расчёте воздействия внешнего импульса.
\end{enumerate}

Решение о необходимости учёта данных эффектов будет принято после завершения серии
натурных экспериментов в интересующем скоростном диапазоне.